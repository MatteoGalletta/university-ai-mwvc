\chapter{Genetic Algorithms}

\section{Behaviour and Structure}

Genetic Algorithms (GAs) are a class of evolutionary algorithms inspired by the principles of natural selection. They operate by iteratively evolving a population of potential solutions towards an optimal or near-optimal state. The process unfolds as follows:

\begin{description}

	\item[Initialization] An initial population of candidate solutions is randomly generated.

	\item[Evaluation] Each individual in the population is evaluated based on a predefined fitness function, which quantifies its quality or suitability with respect to the problem at hand.

	\item[Selection] A subset of individuals are selected from the populations. The chosen individuals are selected as parents of the following population. Selection is based on fitness, with fitter individuals having a higher probability of being chosen.

	\item[Crossover] Using crossover, selected parents genetic material are combined to create offspring.
	
	\item[Mutation] With a certain probability, random mutations are introduced into the offspring. This is called mutation and it helps to maintain diversity within the population and prevents premature convergence towards suboptimal solutions.
	
	\item[Replacement] This newly generated population replaces the old one.
	
\end{description}

The cycles of evaluation, selection, crossover, mutation, and replacement are repeated until a halting criteria is satisfied.

\section{k-Tournament Selection}

The selection algorithm for the proposed solution is \emph{k-tournament}.
This algorithm involves running several "tournaments" among k individuals chosen at random from the population, until the desired amount of population is reached. Each tournament selects the best amongst the k selected individuals.

The tournament size $k$ can be adjusted to balance exploration and exploitation. Smaller k introduces more diversity, while larger k focuses more on exploiting fittest individuals.
Given a population of $n$ individuals:
\begin{enumerate}
  \item[] $k=1:$ \quad selection is random, there is no preference based on fitness
  \item[] $k=n:$ \quad the fittest individuals are always selected 
\end{enumerate}

\begin{framed}
\begin{lstlisting}[caption=k-tournament selection]
function k-tournament(population, k, n)
	new_population = []
	while new_population.size < n
		selected = []
		while selected.size < k
			individual = (*@\textit{random element from population}@*)
			selected.push(individual)
		end
		new_population.push(best(selected))
	end
	return new_population
end
\end{lstlisting}
\end{framed}


\section{Single-Point Crossover}

The simplest form of crossover is the single-point crossover, where a random crossover point is selected and the genetic material is exchanged between the parents at that point.

\begin{framed}
\begin{lstlisting}[caption=Single-point crossover]
function single-point-crossover(parent1, parent2)
	i = (*@\textit{random integer in [0..n-1]}@*)
	offspring1 = parent1[0..i] + parent2[i+1..n-1]
	offspring2 = parent2[0..i] + parent1[i+1..n-1]
	return offspring1, offspring2
end
\end{lstlisting}
\end{framed}

\section{Bit-Flip Mutation}

Mutation is a genetic operator that introduces random changes in the offspring. The simplest form of mutation is the bit-flip mutation, where a bit has a probability $p$ of being flipped.

\begin{framed}
\begin{lstlisting}[caption=Bit-flip mutation]
function bit-flip-mutation(offspring, p)
	for i in 0..n-1
		if random() < p
			offspring[i] = 1 - offspring[i]
		end
	end
	return offspring
end
\end{lstlisting}
\end{framed}






